% Options for packages loaded elsewhere
\PassOptionsToPackage{unicode}{hyperref}
\PassOptionsToPackage{hyphens}{url}
\PassOptionsToPackage{dvipsnames,svgnames,x11names}{xcolor}
%
\documentclass[
  authoryear,
  preprint,
  3p]{elsarticle}

\usepackage{amsmath,amssymb}
\usepackage{iftex}
\ifPDFTeX
  \usepackage[T1]{fontenc}
  \usepackage[utf8]{inputenc}
  \usepackage{textcomp} % provide euro and other symbols
\else % if luatex or xetex
  \usepackage{unicode-math}
  \defaultfontfeatures{Scale=MatchLowercase}
  \defaultfontfeatures[\rmfamily]{Ligatures=TeX,Scale=1}
\fi
\usepackage{lmodern}
\ifPDFTeX\else  
    % xetex/luatex font selection
\fi
% Use upquote if available, for straight quotes in verbatim environments
\IfFileExists{upquote.sty}{\usepackage{upquote}}{}
\IfFileExists{microtype.sty}{% use microtype if available
  \usepackage[]{microtype}
  \UseMicrotypeSet[protrusion]{basicmath} % disable protrusion for tt fonts
}{}
\makeatletter
\@ifundefined{KOMAClassName}{% if non-KOMA class
  \IfFileExists{parskip.sty}{%
    \usepackage{parskip}
  }{% else
    \setlength{\parindent}{0pt}
    \setlength{\parskip}{6pt plus 2pt minus 1pt}}
}{% if KOMA class
  \KOMAoptions{parskip=half}}
\makeatother
\usepackage{xcolor}
\setlength{\emergencystretch}{3em} % prevent overfull lines
\setcounter{secnumdepth}{5}
% Make \paragraph and \subparagraph free-standing
\ifx\paragraph\undefined\else
  \let\oldparagraph\paragraph
  \renewcommand{\paragraph}[1]{\oldparagraph{#1}\mbox{}}
\fi
\ifx\subparagraph\undefined\else
  \let\oldsubparagraph\subparagraph
  \renewcommand{\subparagraph}[1]{\oldsubparagraph{#1}\mbox{}}
\fi


\providecommand{\tightlist}{%
  \setlength{\itemsep}{0pt}\setlength{\parskip}{0pt}}\usepackage{longtable,booktabs,array}
\usepackage{calc} % for calculating minipage widths
% Correct order of tables after \paragraph or \subparagraph
\usepackage{etoolbox}
\makeatletter
\patchcmd\longtable{\par}{\if@noskipsec\mbox{}\fi\par}{}{}
\makeatother
% Allow footnotes in longtable head/foot
\IfFileExists{footnotehyper.sty}{\usepackage{footnotehyper}}{\usepackage{footnote}}
\makesavenoteenv{longtable}
\usepackage{graphicx}
\makeatletter
\def\maxwidth{\ifdim\Gin@nat@width>\linewidth\linewidth\else\Gin@nat@width\fi}
\def\maxheight{\ifdim\Gin@nat@height>\textheight\textheight\else\Gin@nat@height\fi}
\makeatother
% Scale images if necessary, so that they will not overflow the page
% margins by default, and it is still possible to overwrite the defaults
% using explicit options in \includegraphics[width, height, ...]{}
\setkeys{Gin}{width=\maxwidth,height=\maxheight,keepaspectratio}
% Set default figure placement to htbp
\makeatletter
\def\fps@figure{htbp}
\makeatother

\usepackage{booktabs}
\usepackage{longtable}
\usepackage{array}
\usepackage{multirow}
\usepackage{wrapfig}
\usepackage{float}
\usepackage{colortbl}
\usepackage{pdflscape}
\usepackage{tabu}
\usepackage{threeparttable}
\usepackage{threeparttablex}
\usepackage[normalem]{ulem}
\usepackage{makecell}
\usepackage{xcolor}
\usepackage{todonotes,mathtools,bm,amsmath,mathpazo}
\mathtoolsset{showonlyrefs}
\setlength{\parindent}{0cm}
\makeatletter
\@ifpackageloaded{caption}{}{\usepackage{caption}}
\AtBeginDocument{%
\ifdefined\contentsname
  \renewcommand*\contentsname{Table of contents}
\else
  \newcommand\contentsname{Table of contents}
\fi
\ifdefined\listfigurename
  \renewcommand*\listfigurename{List of Figures}
\else
  \newcommand\listfigurename{List of Figures}
\fi
\ifdefined\listtablename
  \renewcommand*\listtablename{List of Tables}
\else
  \newcommand\listtablename{List of Tables}
\fi
\ifdefined\figurename
  \renewcommand*\figurename{Figure}
\else
  \newcommand\figurename{Figure}
\fi
\ifdefined\tablename
  \renewcommand*\tablename{Table}
\else
  \newcommand\tablename{Table}
\fi
}
\@ifpackageloaded{float}{}{\usepackage{float}}
\floatstyle{ruled}
\@ifundefined{c@chapter}{\newfloat{codelisting}{h}{lop}}{\newfloat{codelisting}{h}{lop}[chapter]}
\floatname{codelisting}{Listing}
\newcommand*\listoflistings{\listof{codelisting}{List of Listings}}
\makeatother
\makeatletter
\makeatother
\makeatletter
\@ifpackageloaded{caption}{}{\usepackage{caption}}
\@ifpackageloaded{subcaption}{}{\usepackage{subcaption}}
\makeatother
\journal{International Journal of Production Research}
\ifLuaTeX
  \usepackage{selnolig}  % disable illegal ligatures
\fi
\usepackage[]{natbib}
\bibliographystyle{elsarticle-harv}
\usepackage{bookmark}

\IfFileExists{xurl.sty}{\usepackage{xurl}}{} % add URL line breaks if available
\urlstyle{same} % disable monospaced font for URLs
\hypersetup{
  pdftitle={The missing puzzle piece: Contextual insights for enhanced pharmaceutical supply chain forecasting},
  pdfkeywords={Forecasting, Pharmaceutical supply chain, Domain
knowlege, Forecast accuracy, Developing countries},
  colorlinks=true,
  linkcolor={blue},
  filecolor={Maroon},
  citecolor={Blue},
  urlcolor={Blue},
  pdfcreator={LaTeX via pandoc}}


\begin{document}

\begin{frontmatter}
\title{The missing puzzle piece: Contextual insights for enhanced
pharmaceutical supply chain forecasting}


\cortext[cor1]{Corresponding author}
        
\begin{abstract}
Accurate forecasting in pharmaceutical supply chains is critical for
ensuring the continuous availability of essential medicines,
particularly in developing countries where resource constraints and
logistical challenges are more pronounced. Effective demand forecasting
supports procurement and inventory management, which in turn could help
to prevent stockouts, reduce wastage due to overstocking, and enhance
overall healthcare delivery. In such settings, reliable forecasts, that
acknowledge uncertainties, are essential to ensure that limited
resources are used efficiently to meet health needs. Despite
advancements in forecasting, significant gaps remain in effective
forecasting in poor resource countries. Most existing literature on
pharmaceutical supply chain forecasting focuses primarily on point
estimation of consumption while neglecting the inherent uncertainty of
these forecasts. Additionally, these studies often rely solely on
historical consumption data without considering the broader context that
influences consumption. Furthermore, many works lack rigorous
methodological design, transparency, and reproducibility. This study
addresses these gaps by integrating domain-specific knowledge, gathered
through expert interviews and engagement with key members of the
Ethiopian Pharmaceutical Supply Service (EPSS), into forecasting models.
Using a dataset spanning five years (December 2017 to July 2022) from
EPSS, we developed forecasting models that integrate expert-identified
variables such as stock replenishment schedules, fiscal inventory
counts, and disease outbreaks. Evaluation metrics including Mean
Absolute Scaled Error (MASE), Root Mean Squared Scaled Error (RMSSE),
and Continuous Ranked Probability Score (CRPS) are used to report
forecast accuracy. The findings underscore the significance of
contextual data in developing robust forecasting models that are
adaptable to complex, real-world conditions. Our results also highlight
the effectiveness of foundational time series models for forecasting.
These models are particularly appealing for resource-constrained
countries that may lack advanced analytical expertise. To promote
usability, generalizability, and reproducibility, we share the complete
dataset and code in R and Python, and the entire paper is written in
Quarto via a GitHub repository to facilitate these practices.
\end{abstract}





\begin{keyword}
    Forecasting \sep Pharmaceutical supply chain \sep Domain
knowlege \sep Forecast accuracy \sep 
    Developing countries
\end{keyword}
\end{frontmatter}
    
\section{Introduction}\label{sec-intro}

Universal access to essential medicines is a fundamental goal of
effective healthcare systems, yet ensuring equitable availability
continues to be a critical global challenge
\citep{quick2003essential, world2004annual}. Many countries,
particularly in Africa and Asia, face significant barriers to accessing
these medications, often due to high costs and inefficient supply chain
management \citep{world2004medicines}. A survey conducted in eight
sub-Saharan African countries revealed alarmingly low availability of
essential medicines. The average availability of women's priority
medicines ranged from just 22\% to 40\%, while children's medicines were
only slightly more accessible, with availability ranging from 28\% to
57\% \citep{droti2019poor}. However, drug shortages are not limited to
low-income regions; they also occur in developed areas like the United
States and Europe, where they disrupt healthcare delivery and compromise
the quality of patient care
\citep{fox2003managing, kaakeh2011impact, johnson2011drug, huys2013european, le2011prevalence}.

The consequences of drug shortages extend beyond immediate healthcare
delivery, impacting healthcare costs, treatment quality, and patient
safety \citep{alspach2012drug, kaakeh2011impact, baumer2004national}.
Such shortages can result in treatment delays, suboptimal care due to
the unavailability of preferred therapies, increased reliance on costly
secondary markets, and serious patient outcomes, including
complications, longer hospital stays, and even deaths
\citep{alspach2012drug}. Addressing these challenges requires a
resilient pharmaceutical supply chain that can ensure the timely
availability of medicines. In this context, accurate demand forecasting
becomes critical for effective procurement and inventory management
\citep{subramanian2021effective}.

Demand forecasting in pharmaceutical supply chains is a highly complex
process, shaped by several factors. These include the quality of
available data, the specific position within the supply chain, and
external influences such as market trends, regulatory changes, and
disruptive events such as pandemics, conflicts, flooding
\citep{schneider2010pharmaceutical, hyndman2018forecasting}. Recently,
major donors such as Bill \& Melinda Gates Foundation have invested in
data collection technologies in sub-Saharan Africa, recognizing the
potential of data to provide valuable insights. This investment is
essential and offers significant benefits for improving supply chain
delivery. However, much of the data being collected focuses on
transactional data, such as consumption and sale. Almost all platforms
collecting data tend to overlook the critical contextual factors and
events that influence these transactions---factors that are just as
crucial for accurate forecasting. This presents a major barrier, as
understanding the broader context, including administration procedures,
local policies, disruptions, conflicts, and so on is vital for building
reliable forecasting models. This gap---known as business context and
domain knowledge---requires urgent attention in Pharmaceutical supply
chain forecasting. Without incorporating this knowledge, any forecasting
model is unlikely to produce reliable results. We must prioritize
collecting and integrating this contextual information to improve the
accuracy of demand forecasts in pharmaceutical supply chains. Combining
statistical models with domain insights offers a more refined
forecasting process, as evidenced by research that highlights the value
of judgmental adjustments to enhance forecast accuracy
\citep{fildes2007against, taylor2022artificial, soyiri2013overview}.
Domain knowledge enables analysts to incorporate relevant events, such
as supply chain disruptions, into the forecasting model. This
integration is particularly beneficial in fields like healthcare, where
expert insights can align algorithms with real-world applications
\citep{dash2022review}.

This study aims to bridge that gap by employing advanced machine
learning models combined with domain expertise to address the
distinctive characteristics of Ethiopia's pharmaceutical sector. The
challenges in this sector, such as diverse product categories, limited
data from service delivery points, communication issues, long lead
times, forecast inaccuracies, and complex policies, necessitate tailored
forecasting approaches
\citep{bilal2024challenges, boche2022procurement}. Accurate demand
forecasting in Ethiopia is essential for optimizing resource allocation,
streamlining supply chain operations, and shaping effective healthcare
policies \citep{rostami2022forecasting}.

This research uses consumption data from 33 health commodities, covering
the period from December 2017 to July 2022, provided by the Ethiopian
Pharmaceutical Supply Service (EPSS), to develop models for
six-month-ahead forecasting. Initially, we implement univariate models
that rely solely on consumption data. Following this, we conduct
exploratory data analysis and engage in interviews with EPSS experts to
gather domain knowledge on events influencing consumption patterns.
Through these interviews, the experts identified three key factors
collectively impacting consumption: stock replenishment period, fiscal
calendar, and the seasonal malaria. We then develop forecasting models
that incorporate the data gathered through the interview process and
compare their performance with univariate models that rely solely on
consumption data.

Our primary aims is to assess whether incorporating domain knowledge
improves forecast accuracy. Moreover, we also fill a gap in the
pharmaceutical demand forecasting literature by not only generating
point forecasts but also producing probabilistic forecasting techniques
that capture the range of potential outcomes with associated
probabilities, offering a more comprehensive view of demand uncertainty
than traditional methods. Model performance is evaluated using point
accuracy measure including MASE, RMSSE, and probabilistic measure using
CRPS.

The remainder of this paper is structured as follows: Section 2 reviews
the literature and identifies gaps to position our work; Section 3
outlines the experimental design, including data sources, forecasting
methods, and evaluation criteria; Section 4 presents the results and
analysis; and Section 5 concludes with a summary of findings and future
research directions.

\section{Research background}\label{sec-lit}

Accurate sales forecasting is crucial in the pharmaceutical industry,
where it directly impacts profit maximization, cost minimization, and
the ability to respond to market changes. Forecasting in healthcare not
only influences clinical decisions but also plays a pivotal role in
supply chain management, ensuring that the right drugs are available
when needed. The challenge lies in balancing customer demand with
inventory costs, which is particularly complex in pharmaceuticals due to
factors like short shelf life and quality constraints
\citep{gupta2000mid, makridakis2020forecasting}.

In this review, a range of forecasting methods has been applied, from
traditional statistical models to advanced machine learning techniques,
with many studies using a combination of simple and advanced methods to
enhance accuracy
\citep{nikolopoulos2016forecasting, zhu2021demand, anusha2014demand}.
Some studies, such as those by \citet{newberne2006holt}, and
\citet{restyana2021analysis}, applied traditional forecasting methods
like the Holt-Winters method, Single Moving Average (SMA), and Single
Exponential Smoothing (SES), which are simpler but effective for certain
applications. In contrast, other researchers explored more advanced
approaches, including hybrid models and machine learning techniques, to
address complex forecasting challenges, as seen in the studies by
\citet{siddiqui2022hybrid}, \citet{de2021lead}, \citet{kim2015demand},
\citet{candan2014demand}, and \citet{ribeiro2017improving}.

Early research in pharmaceutical forecasting primarily focused on
traditional statistical methods. These methods, including ARIMA,
exponential smoothing, and moving averages, were widely used due to
their simplicity and effectiveness in relatively stable environments
\citep{zahra2019forecasting}. These models typically assess forecast
accuracy using metrics like Mean Absolute Percentage Error (MAPE), Mean
Squared Error (MSE), and Root Mean Squared Error (RMSE). For example,
\citet{newberne2006holt} applied the Holt-Winters method to forecast
healthcare data, focusing on prescription trends. Their findings
highlighted the method's utility in short-term planning and resource
management, validated through metrics like MAPE, MAD, and MSD.
Similarly, \citet{restyana2021analysis} compared Single Moving Average
(SMA) and Single Exponential Smoothing (SES) in drug demand forecasting,
concluding that SES provided more accurate results, as indicated by
lower MAD and MSE values.

Recognizing the limitations of traditional models, particularly in
capturing the complexities of pharmaceutical demand (e.g., seasonality,
external influences), researchers began developing hybrid models. These
models combine linear methods like ARIMA with nonlinear techniques such
as neural networks, aiming to improve forecasting accuracy by addressing
diverse data patterns. \citet{khalil2014intelligent} introduced a hybrid
model that integrated ARIMA with neural networks. Their study
demonstrated that this approach significantly outperformed traditional
models, especially in scenarios with limited historical data. The hybrid
model's effectiveness was confirmed by improvements in key performance
indicators like MAPE and RMSE.

The pharmaceutical industry's increasing complexity and the availability
of large datasets have led to the adoption of advanced machine learning
techniques. These methods, such as Support Vector Regression (SVR),
Random Forest (RF), and Long Short-Term Memory (LSTM) networks, offer
greater flexibility and predictive power. \citet{van2021using} showed
that incorporating downstream information into machine learning models,
like LASSO and SVR, greatly enhances forecast accuracy. Their study on
multi-echelon supply chains revealed that advanced models, particularly
those integrating external variables, consistently produced lower
forecast errors, as measured by AvgRelRMSE. Similarly,
\citet{rathipriya2023demand} compared shallow neural networks and deep
learning models for drug demand forecasting. The study found that
shallow models outperformed deep learning models for most drug
categories, while ARIMA was more effective for the remaining categories.
This suggests that no single model is universally optimal, and model
selection should be context-specific.

More sophisticated models, such as Neuro-Fuzzy Systems, have emerged as
powerful tools for pharmaceutical forecasting. These systems combine the
learning capabilities of neural networks with the reasoning abilities of
fuzzy logic, offering a balanced approach that incorporates both
empirical data and expert knowledge. Deep learning models, while
powerful, present challenges such as the need for large datasets and
substantial computational resources. Despite these challenges, models
like LSTM have shown potential in capturing long-term dependencies in
time series data, as demonstrated by \citet{sousa2019statistical} in
their study on drug distribution in the Brazilian Public Health System.
\citet{candan2014demand} employed an Adaptive Neuro-Fuzzy Inference
System (ANFIS) to forecast pharmaceutical demand. Their study
highlighted the system's ability to capture complex patterns in the
data, resulting in highly accurate forecasts. The effectiveness of this
approach was further validated through statistical tests, such as paired
T-tests and mean difference analysis, which confirmed its superiority
over traditional methods.

\citet{nguyen2023managing} explore how sentiment analysis of news media
can be utilized to improve demand forecasting for pharmaceutical
products during disruptive events, such as pandemics and scandals. The
authors employed a VARX (Vector Autoregressive with Exogenous Variables)
model to forecast demand volatility. \citet{papanagnou2018coping}
investigate the impact of big data analytics, specifically through text
mining, on improving forecasting accuracy for pharmaceuticals in retail
pharmacies, using HMX pharmacy stores in Nigeria as a case study. They
use multivariate time series analysis technique known as VARX, which
incorporates customer-generated content from sources like Google,
YouTube, and online newspapers to predict demand for medicines.
\citet{fourkiotis2024applying} focus on enhancing pharmaceutical sales
forecasting through the application of both traditional statistical
methods and advanced machine learning techniques. Various forecasting
approaches, including traditional models like Naïve and ARIMA,Facebook
Prophet, and machine learning methods such as XGBoost and LSTM neural
networks are compared. The results demonstrated that machine learning
models, particularly XGBoost, significantly outperformed traditional
methods. \citet{belghith2024new} present a rolling forecasting framework
employing moving average and three exponential smoothing methods
(Brown's, Holt's, and Holt-Winters) implemented using Microsoft Power BI
for enhanced data visualization.

Table~\ref{tbl-literature} provides a summary of key studies in the
literature on forecasting for pharmaceutical products. We identify
several limitations both in the existing research and in practice, which
highlight important gaps that motivate our current study. These gaps are
summarized as follows:

\begin{enumerate}
\def\labelenumi{\arabic{enumi}.}
\item
  Despite significant investment in data collection technologies and
  logistics management systems, these systems primarily collect
  transactional data, such as consumption. However, no data is collected
  on events that influence consumption variability---information that
  would be crucial for building reliable forecasting models. No studies
  to date have explored how domain expertise could enhance
  pharmaceutical demand forecasting.
\item
  Current research on forecasting for pharmaceutical supply chain
  predominantly focuses on generating point forecasts. There is a lack
  of studies that consider the entire forecast distribution of monthly
  consumption, which would better capture the uncertainty of future
  demand and provide a valuable risk management tool for
  decision-makers.
\item
  Reproducibility remains a major challenge in this field. It is often
  difficult for readers to reproduce previous studies without direct
  assistance from the authors, limiting the practical application and
  validation of existing research.
\end{enumerate}

\begin{table}[H]

\caption{\label{tbl-literature}Summary of some studies on forecasting in
pharmaceutical supply chain}

\centering{

\centering\begingroup\fontsize{11}{13}\selectfont

\resizebox{\linewidth}{!}{
\begin{tabular}{lll>{\raggedright\arraybackslash}p{10em}ll>{\raggedright\arraybackslash}p{10em}>{\raggedright\arraybackslash}p{10em}>{\raggedright\arraybackslash}p{2em}>{\raggedright\arraybackslash}p{2em}>{\raggedright\arraybackslash}p{2em}}
\toprule
References & data granularity & Lenth of data & Forecast variable & Horizon & Probabilistic & Method & Metric & Items & Data & Code\\
\midrule
Current study & Monthly & 60 months & Consumption of pharmaceutical products & 6 months & Yes & TimeGpt (with and without predictors), LSTM (with and without predictors), Exponential Smoothing, ARIMA, Dynamic Regression (ARIMAX), Multiple Regression (with and without predictors) & RMSSE,CRIPS,MASE & 33 & Yes & Yes\\
Siddiqui et al. (2022) & Monthly & 58 months & Sales of pharmaceutical products & Un known & No & ARIMA, HOLT WINTER, ETS,  Theta, ARHOW & MEAN, MAD, MSE, RMSE,  MAPE & 31 & No & No\\
de Oliveira et al.(2021) & Daily & 365 days & Lead time & NA & No & k-NN, SVM, RF, LR, MLP & MSE & 11 & Yes & No\\
Rathipriya et al. (2023) & Weekly & 260 weeks & Sales of pharmaceuticals product & Un known & No & SNN, ARIMA P\_NN GR\_NN RBF\_NN ARIMA P\_NN GR\_NN RBF\_NN & RMSE, Normalized RMSE & 57 & Yes & No\\
Kim et al.(2015) & Monthly & 44 month & Medicines sales data & 1 month & No & VARX & Prediction error rate & 4 & No & No\\
Merkuryeva et al. (2019) & Weekly & 20 weeks & Sales of pharmaceuticals & 1–week & No & MA, Symbolic Regression, LR & R², MAD & 1 & No & No\\
Nikolopoulos et al. (2016) & Yearly & 21 year & Prescription records of pharmaceuticals & 1–5 & No & Diffusion models,  ARIMA, ES, LR & R², ME, MAE, MSE & 14 & No & No\\
Van Belle et al.(2021) & Weekly & 253 weeks & Medcine sales & 1–5 weeks & No & ARIMA, ETS, MLP, SVR, RF, LR & RMSE & 50 & No & No\\
Khalil Zadeh et al. (2014) & Monthly & 36 months & Sales  pharmaceutical products & 1 month & No & ARIMA, Graph-based analysis and ANN, & R2,MSE, MAE & 217 & No & No\\
Zhu  X et al.(2021) & Weekly & 520 weeks & Medcine sales & 1–2 months & No & MA, LR, Clustering and RNN & NME, NMAE, NMSE & 245 & No & No\\
Anusha et al.(2014) & Monthly & 36 months & Sales of pharmaceuticals & Un known & No & 6 Month MA, SES, WES & MAD, MSE, MAPE & 2 & Yes & No\\
Burinskiene (2022) & Weekly & 13 weeks & Medcine sales & 36 days & No & SES, MA, Naïve, Holt & MAPE,MSE, U² & 8 & No & No\\
Candan et al. (2014) & Yearly & 6 years & Medcine sales & 3 months & No & NF & SE Mean & 1 & No & No\\
Ribeiro et al.(2017) & Monthly & 24 months & Medcine sales & 3 months & No & DPM & SMAPE & 357 & No & No\\
Bon and NG (2017) & Monthly & 68 months & Medcine sales & 1 month & No & SMA, SES, DMA, DES, RA, HW-A, SA, HW-M, ARIMA-SM & RMSE, MSE, MAD,  MAPE  & 1 & No & No\\
Newberne and Mike (2006)  & Monthly & 36 months & Pseudoephedrine prescriptions & 3 months & No & HW-M & MAPE & 1 & No & No\\
Sousa et al.(2019) & Monthly & 79 months & drug distribution & quarterly & No & Naïve, Seasonal Naïve, ARIMA, DLM, LST, Recurrent Neural Network, SES, Holts linear and damped trends, HW, ES-SSM, Theta & MAE & 30 & Yes & Yes\\
Zahra and Putra (2019) & Monthly & 36 months & Consumption & 12 months & No & ARIMA, ES & RMSE, MAPE, MAD, MSE & 1 & No & No\\
Nguyen et  al. (2023) & Un known & Un known & Pharmaceutical products & Un known & No & VARX & ME, MAE,MAPE,RMSE & Unknown & No & No\\
Christos et al.(2017) & Weekly & 129 weeks & Web-scraping  of analgesic medcines & Un known & Yes & VARX  & AIC, BIC, MSE, MAE & Unknown & No & No\\
Belghith et al.(2024) & Daily & 1095 days & Sales of pharmaceuticals & 1 year & No & Brown’s method, Holt’s method, and Holt-Winters method, moving average & MAD, MAPE & 125 & No & No\\
Konstantinos  et al.(2024) & Weekly & 260 weeks & Medcine sales & 1 year & No & Naïve, ARIMA, XGBoost , LSTM, Prophet & MAPE,MSE & 57 & No & No\\
\bottomrule
\end{tabular}}
\endgroup{}

}

\end{table}%

\section{Experiment setup}\label{sec-experiment}

\subsection{Data}\label{sec-data}

For this study, we use a dataset spanning five years (Dec.~2017- July
2022) of consumption data obtained from EPSS. Rigorous checks were
conducted to ensure the consistency and completeness of the collected
data. From the extensive pool of pharmaceutical products within EPSS, we
selected a set of 33 key pharmaceuticals, including various programs and
representing different classes of drugs.

\subsubsection{Explaoratory Data Analysis with consumption
data}\label{explaoratory-data-analysis-with-consumption-data}

Given the number of products in this study, we created several data plot
and computed features of the time series, including the strength of
trend and seasonality to better understadn data.
Figure~\ref{fig-feature} shows the strength of trend versus seasonality.
Each point represents one time series with the strength of trend in
x-axis and the strength of seasonality in y-axis. Both measures are on a
scale of {[}0,1{]}. the strength of trend and seasonality were
calculated using the ``STL'' (Seasonal and Trend decomposition using
Loess) decomposition method, as described by \citet{mstl}.

\begin{figure}

\centering{

\includegraphics[width=0.6\textwidth,height=\textheight]{main_files/figure-pdf/fig-feature-1.pdf}

}

\caption{\label{fig-feature}The strength of the trend and seasonality in
the time series of pharmaceutical product consumption. The scatter plot
shows 33 data points, with each point corresponding to a product.}

\end{figure}%

It is evident that some time series display strong trends and/or
seasonality, while the majority exhibit low trends and minimal
seasonality. A number of products show pronounced trends, and only a few
demonstrate clear seasonal patterns. Beyond assessing the strength of
trends and seasonality, we also visualized all time series to understand
data and various patterns, including trends and instances of erratic
consumption behavior during certain months. For example, some series
show low consumption volumes over consecutive months, followed by peak
consumption in specific months, making them more volatile and difficult
to forecast. This underscores the diversity of monthly pharmaceutical
time series patterns within the dataset and highlights the importance of
understanding the factors driving these consumption behaviors. Figure
Figure~\ref{fig-dataviz2} illustrates the time plots for a few selected
products.

\begin{figure}

\centering{

\includegraphics{main_files/figure-pdf/fig-dataviz2-1.pdf}

}

\caption{\label{fig-dataviz2}Monthly time plot of consumption. X-axis
shows the month, consisting of 60 data points (months) and y-axis shows
the comsumption. The panels display data from four products to give a
glimpse of the consumption patterns.}

\end{figure}%

Visualizing the data revealed that various events significantly impacted
the consumption of different products, but these were not reflected in
the available data. Expert insights were crucial for understanding the
nature and timing of these events, filling gaps in the system, and
incorporating qualitative factors that EPSS logistics system miss.
Therefore, we conducted interviews with domain experts to collect
information on external factors that influence consumption. These
interviews allowed us to account for unusual patterns and customize our
forecasting models to more accurately reflect the unique consumption
behaviors of each product.

\subsubsection{Collaborative expert review to identify factors affecting
consumption}\label{collaborative-expert-review-to-identify-factors-affecting-consumption}

To understand the external factors influencing product consumption, we
began by visualizing the time series data for consumption as highlighted
in the previous section. This initial step helped identify specific
instances within the data that showed unusual patterns, such as peaks,
unusually high or low observations, and consecutive periods of low
consumption. These observations indicated potential events or factors
affecting consumption levels.

To gather domain knowledge about factors affected the consumption of
products, we contacted experts from EPSS who work in the distribution
unit. Next, we collaborated with six experts, each with over 10 years of
experience managing pharmaceutical distribution. In these sessions, we
reviewed the time series of all products together to understand the
potential drivers behind these anomalies. The experts provided insights
based on their extensive experience and knowledge of distribution
practices. When there was uncertainty or gaps in the data for certain
products, additional experts were consulted to gather further context.
If all experts reached a consensus, the influencing factors were
confirmed and included in the analysis.

As part of this process, we examined the electronic bin cards for 33
pharmaceutical products and conducted detailed data exploration. A bin
card is a physical or electronic record used to track the inventory
levels of individual items in a storage location, such as a warehouse or
stockroom. The term `bin' refers to the specific storage location for an
item. Information typically recorded on a bin card includes the Item ID,
unit of measure, product description, account details, date, references,
beginning balance, manufacturer, batch number, expiry date, quantity
received, quantity issued, and balance. In summary, a bin card is a
fundamental tool in inventory management, especially in environments
where physical stock tracking is essential. It provides a
straightforward and immediate way to monitor item quantities, helping to
ensure effective inventory management. Over a period of two weeks, we
meticulously traced consumption data through these bin cards. Moreover,
We ensured accuracy by converting dates from the Ethiopian calendar to
the Gregorian calendar and excluded transactions related to internal
storage or warehouse relocation within EPSS. Through this collaborative
approach, we were able to identify several predictors affecting
distribution and consumption that are summarised as followings:

\begin{itemize}
\tightlist
\item
  Stock replenishment: refers to the process of restocking or refilling
  inventory to ensure that there are sufficient quantities of products
  or materials available to meet demand. Whenever there was stock
  replenishment at the central EPSS, consumption and distribution to
  hubs and health facilities increased. This increase was attributed to
  the need to restock depleted inventories and the push from central
  EPSS to manage space constraints.
\end{itemize}

\begin{itemize}
\tightlist
\item
  Physical fiscal year inventory counting: refers to the process of
  manually counting and verifying the actual quantities of
  pharmaceutical products available in stock at a specific location. The
  process is critical for maintaining the accuracy of inventory records,
  ensuring that medicines are available when needed, and preventing
  stockouts or overstocking. Physical inventory counting periods also
  influenced consumption. Stores closed during these periods, halting
  transactions. We observed increased consumption before inventory
  counting periods, as hubs and facilities stocked up. July and August
  were identified as physical counting periods each year.
\end{itemize}

\begin{itemize}
\tightlist
\item
  Malaria seasonality: Refers to the predictable patterns and
  fluctuations in malaria incidence throughout the year, typically
  influenced by climate and environmental conditions. In many regions,
  malaria transmission peaks during and shortly after the rainy season,
  when conditions such as stagnant water pools create ideal breeding
  sites for the Anopheles mosquitoes that transmit the disease.
  Conversely, malaria cases often decline during the dry season when
  mosquito breeding sites are reduced. During peak malaria seasons,
  there is a significant surge in the demand for antimalarial drugs and
  other related treatments. Malaria seasonality was another significant
  predictor. Certain pharmaceuticals, like Artemether + Lumefanthrine
  and Rapid Diagnostic Test kits, were affected by malaria outbreaks. We
  identified epidemic periods affecting consumption: September to
  December 2017, March to May 2018, September to December 2018, March to
  May 2019, September to December 2019, March to May 2020, September to
  December 2020, March to May 2021, September to December 2021, and
  March to May 2022.
\end{itemize}

The key predictors identified are deterministic values, since both past
and future values of these predictors are known in advance. We then
integrated them into the consumption data, creating the complete dataset
required for model building and running the experiment.

\subsection{Forecasting models}\label{forecasting-models}

We evaluate a range of univariate models and their counterparts that
include predictors, spanning from simpler methods like regression,
exponential smoothing, and ARIMA to more complex models such as long
short-term memory (LSTM) networks and advanced foundational time series
forecasting models. Below, we provide a brief overview of these
approaches. Detailed implementation codes in R and Python are available
in a GitHub repository and accessible for public.

\textbf{Exponential Smoothing State Space model (ETS):} ETS models, as
described by \citet{hyndman2021forecasting}, combine trend, seasonality,
and error components in time series using different configurations that
can be additive, multiplicative, or mixed. The trend component can be
specified as none (``N''), additive (``A''), or damped additive
(``Ad''); the seasonality can be none (``N''), additive (``A''), or
multiplicative (``M''); and the error term can be additive (``A'') or
multiplicative (``M''). To forecast consumption, we use the ETS()
function from the fable package in R, which automatically selects the
optimal model for each time series based on the corrected Akaike's
Information Criterion (AICc). In our study, an automated algorithm
determines the best configuration for trend, seasonality, and error
components across each time series, leveraging the ets() function's use
of AIC to identify the optimal model. Given the high volume of time
series (1530), manual selection of components is impractical, so the
algorithm customizes model forms for each series based on its unique
characteristics. This results in a tailored combination of additive or
multiplicative components, adapting to the specific patterns of each
time series.

\textbf{Multiple Linear Regression (MLR)}: We use Multiple linear
regression to model the relationship between a consumption and potential
variables influencing its variation. In our first model, we use multiple
linear regression with a trend component to capture the underlying
progression over time, i.e., \emph{regression}. We also incorporate
dummy variables for each month to account for seasonal fluctuations,
without including any additional predictors, i.e.,
\emph{regression\_reg}. This approach helps us establish a baseline
model focused solely on temporal trends and seasonal patterns. We then
extend this model by introducing additional predictors that include
variables such as replenishment cycles, fiscal year indicators, and
periods with malaria prevalence. These additional predictors allow the
model to see if capturing external factors can provide a better
understanding of the factors influencing the consumption and result at
enhanced accuracy. We produce forecasts using TLSM() function from the
fable package in R.

\textbf{ARIMA and ARIMA with regressors}: ARIMA (AutoRegressive
Integrated Moving Average) is a powerful statistical model designed to
forecast time series by capturing temporal dynamics and are widely used
in time series forecasting due to their ability to model complex trends
and patterns over time without relying on external predictors. ARIMAX
(AutoRegressive Integrated Moving Average with eXogenous variables)
extends ARIMA by incorporating external variables, or exogenous
predictors, into the model. This modification allows to include relevant
information from external factors such as malaria season, and fiscal
year period, and stock replenishement period that may explain variations
in the series beyond its internal time dynamics, we refer to this in the
result as \emph{arima\_reg}. By adding these predictors, ARIMAX combines
the strengths of ARIMA's time-series structure with the flexibility of
regression models. In our study, we use an automated algorithm to
determine the optimal configuration for ARIMA components, following the
approach outlined by \citet{hyndman2021forecasting} and We use the
ARIMA() function from the fable package in R.

\textbf{Long Short-Term Memory neural network (LSTM)}: The LSTM model is
a specialised form of recurrent neural network (RNN) used to model
sequential data by capturing long-term dependencies
\citep{graves2012long}. Unlike traditional RNNs, LSTMs can learn to
retain information for longer time periods due to their unique
architecture, which consists of several gates that control the flow of
information. This makes LSTMs particularly effective for time series
forecasting. In our implementation, we used a sequential model
architecture, comprising one LSTM layer with 50 units, followed by dense
layers, with the final output being a single linear unit. The Adam
optimizer was employed to minimize the mean squared error, and the model
was trained for 100 epochs using the keras\_model\_sequential() function
from the Keras package in R. We use LSTM models both with and without
predictors, referring to the LSTM model with predictors as
\emph{lstm\_reg}.

\textbf{TimeGPT}: TimeGPT is the first pre-trained foundational model
designed specifically for time series forecasting, developed by Nixtla
\citep{garza2023timegpt}. It uses a transformer-based architecture with
an encoder-decoder configuration but differs from other models in that
it is not based on large language models. Instead, it is built from the
ground up to handle time series data. TimeGPT was trained on more than
100 billion data points, drawing from publicly available time series
across various sectors, such as retail, healthcare, transportation,
demographics, energy, banking, and web traffic. This wide range of data
sources, each with unique temporal patterns, enables the model to manage
diverse time series characteristics effectively. Furthermore, TimeGPT
supports the inclusion of external regressors in its forecasts and can
generate quantile forecasts, providing reliable uncertainty estimation.
We use TimeGpt models both with and without predictors, referring to the
model with predictors as \emph{timegpt\_reg}.

In all models, probabilistic forecasts were generated using
bootstrapping to create 1,000 possible future scenarios for each period
within the forecast horizon \citep{hyndman2021forecasting}.

\subsection{Performance evaluation}\label{performance-evaluation}

To assess the performance of our forecasting methods, we split the data
into a series of training and test sets and apply time series
cross-validation with a forecast horizon of 6 months. Each training set
is expanded in monthly increments, allowing model development and
hyperparameter tuning to be performed on the training data, with errors
evaluated on the corresponding test sets. We assess forecasting
performance using both point and probabilistic error measures.

The error metrics presented here consider a forecasting horizon denoted
by by \(j\), which represents the number of time periods ahead we are
predicting, with \(j\) ranging from 1 to 6 months in our study. Point
forecast accuracy is measured using the Mean Squared Scaled Error (MSSE)
and the Mean Absolute Scaled Error (MASE). The Mean Absolute Scaled
Error (MASE) \citep{HK06, hyndman2021forecasting} is calculated as
follows:

\[
  \text{MASE} = \text{mean}(|q_{j}|),
\] where \[
  q_{j} = \frac{ e_{j}}
 {\displaystyle\frac{1}{T-m}\sum_{t=m+1}^T |y_{t}-y_{t-m}|},
\] and \(e_{j}\) is the point forecast error for forecast horizon \(j\),
\(m = 12\) (as we have monthly seasonal series), \(y_t\) is the
observation for period \(t\), and \(T\) is the sample size (the number
of observations used for training the forecasting model). The
denominator is the mean absolute error of the seasonal naive method in
the fitting sample of \(T\) observations and is used to scale the error.
Smaller MASE values suggest more accurate forecasts. Note that the
measure is scale-independent, thus allowing us to average the results
across series.

Here, \(e_{j}\) represents the point forecast error for forecast horizon
\(j\), with \(m = 12\) (since we are dealing with monthly seasonal
data). The term \(y_t\) denotes the observation at time \(t\), and \(T\)
is the sample size, or the number of observations used for training the
forecasting model. The denominator in the MASE formula is the mean
absolute error of the seasonal naive method over the training sample of
\(T\) observations, providing a basis for scaling the forecast error.
Lower MASE values indicate more accurate forecasts. Notably, this
measure is scale-independent, allowing us to average results across
different series for broader performance comparison.

A related measure is MSSE
\citep{hyndman2021forecasting, makridakis2022m5}, which uses squared
errors rather than absolute errors:
\begin{equation}\phantomsection\label{eq-RMSSE}{
  \text{MSSE} = \text{mean}(q_{j}^2),
}\end{equation} where, \[
  q^2_{j} = \frac{ e^2_{j}}
 {\displaystyle\frac{1}{T-m}\sum_{t=m+1}^T (y_{t}-y_{t-m})^2},
\] Again, this is scale-independent, and smaller MSSE values suggest
more accurate forecasts.

To measure the forecast distribution accuracy, we calculate the
Continuous Rank Probability Score \citep{hyndman2021forecasting}. It
rewards sharpness and penalizes miscalibration, so it measures overall
performance of the forecast distribution.
\begin{equation}\phantomsection\label{eq-CRPS}{
  \text{CRPS} = \text{mean}(p_j),
}\end{equation} where \[
  p_j = \int_{-\infty}^{\infty} \left(G_j(x) - F_j(x)\right)^2dx,
\] where \(G_j(x)\) is the forecasted probability distribution function
for forecast horizon \(j\), and \(F_j(x)\) is the true probability
distribution function for the same period.

Calibration refers to the statistical consistency between the
distributional forecasts and the observations. It measures how well the
predicted probabilities match the observations. On the other hand,
sharpness refers to the concentration of the forecast distributions ---
a sharp forecast distribution results in narrow prediction intervals,
indicating high confidence in the forecast. A model is well-calibrated
if the predicted probabilities match the distribution of the
observations, and it is sharp if it is confident in its predictions. The
CRPS rewards sharpness and calibration by assigning lower scores to
forecasts with sharper distributions, and to forecasts that are
well-calibrated. Thus, it is a metric that combines both sharpness and
miscalibration into a single score, making it a useful tool for
evaluating the performance of probabilistic forecasts.

CRPS can be considered an average of all possible quantiles
\citep[Section 5.9]{hyndman2021forecasting}, and thus provides an
evaluation of all possible prediction intervals or quantiles. A specific
prediction interval could be evaluated using a Winkler score, if
required.

\section{Results and discussion}\label{sec-results}

In this section, we compare the forecasting performance of various
approaches, examining models that incorporate expert-identified business
context predictors versus those that rely solely on historical
consumption data. Point forecast performance is reported using MASE and
RMSSE, while probabilistic forecast accuracy is reported using CRPS.

The forecasting performance is reported in Figure~\ref{fig-point}, in
which the average forecast accuracy over forecast horizon and across all
products is calculated for each origin. We report the distribution of
accuracy metrics across all rolling origins. This shows how models
varies in providing accuracy across different origins. The y-axis
displays models sorted by their MASE and RMSSE values, with the model
exhibiting the lowest error positioned at the bottom. This model is the
Ariam model, which incorporates exogenous variables. Additionally,
Figure~\ref{fig-point} indicates that predictors obtained through
interactions with domain experts enhance point forecast accuracy across
most models. This underscores the critical importance of systematically
collecting such expert-informed data alongside transactional consumption
data.

\begin{figure}[H]

\centering{

\includegraphics{main_files/figure-pdf/fig-point-1.pdf}

}

\caption{\label{fig-point}Distribution of point forecast accuracy across
different origins, averaged across the forecat horizon and all products.
The total number of months used to calculate the accuracy in the test
set is 12 months for each product. MASE and MSSE are relative to the
corresponding values for the training set.}

\end{figure}%

\begin{figure}[H]

\centering{

\includegraphics{main_files/figure-pdf/fig-prob-1.pdf}

}

\caption{\label{fig-prob}Distribution of probabilistic forecast accuracy
across different origins, averaged across the forecat horizon and all
products. The total number of months used to calculate the accuracy in
the test set is 12 months for each product.}

\end{figure}%

Figure~\ref{fig-prob} illustrates the forecast distribution accuracy
measured by CRPS, which evaluates both forecasting calibration and
interval sharpness. A smaller CRPS value indicates better overall
performance. We observed that incorporating domain knowledge improved
forecast accuracy for most models, enhancing not only point forecasts
but also probabilistic forecasts. Notably, ARIMA achieved the highest
probabilistic forecast accuracy, while LSTM models had the lowest. This
aligns with the findings related to point forecast accuracy, reinforcing
the earlier explanations for these results.

We observe that LSTM models exhibit a widespread error distribution
compared to other models. This is attributed to their performance
variability, which can significantly depend on the characteristics of
demand patterns across different products. For example, the product
``Amlodipine - 5mg - Tablet'' demonstrates periods of extreme
variability, with spikes in demand followed by periods of very low or
zero demand. Such patterns align well with the strengths of LSTM models,
which excel at capturing long-term dependencies and managing complex
temporal fluctuations. In contrast, the demand for ``Anti-Rho (D)'' is
erratic and sporadic, characterized by frequent, random fluctuations
without any discernible long-term trends or seasonal patterns. This
chaotic nature, combined with inconsistent and often low volumes, poses
significant challenges for LSTM models. The lack of clear structure or
stable patterns limits the models' ability to generalize effectively,
resulting in poorer forecast performance. Although LSTM models can
handle irregular data to some extent, they require sufficient and
well-distributed training samples to achieve optimal performance. In our
study, the majority of products exhibit erratic and sporadic demand
patterns similar to ``Anti-Rho (D)'', resulting in this widespread error
distribution.

The findings emphasize the importance of incorporating relevant domain
knowledge into forecasting models. In pharmaceutical supply logistics
administration systems, data often consists solely of transactional
records on consumption and distribution. Including exogenous variables
such as administrative procedures, seasonal patterns, or conflicts, or
any other relevant factor that could be identified by those with domain
knowledge proved essential for reducing forecast errors. This approach
improved both point and probabilistic forecast accuracy, enabling a more
confident assessment of uncertainty. Therefore, the systematic
collection of information about significant events and their impact on
consumption is vital. Recording details of events such as policy
changes, administration procedures, conflicts, and incorporating this
information into forecasting models creates a comprehensive
understanding of consumption. This practice enhances modeling
reliability and allows institutions in developing countries and
humanitarian organizations to better forecast demand, allocate resources
effectively, and respond proactively. Establishing robust data
collection systems is thus a critical step for strengthening forecasting
capabilities and operational resilience.

Our analysis also highlights the value of using foundational time series
forecasting models like TimeGpt, particularly in developing countries
and the global health and humanitarian sectors. In these contexts, the
lack of advanced analytical skills and systematic data collection can
hinder effective forecasting and decision-making.

\subsection{An illustration of probabilistic forecast for Pharmacuitical
product
consumption}\label{an-illustration-of-probabilistic-forecast-for-pharmacuitical-product-consumption}

In this section, we present an illustrative example of a probabilistic
forecast for future consumption of \emph{Sodium Chloride (Normal
Saline)} product. Due to the complexity of including such visualizations
for all products, only one example is shown here. However, it is
feasible to generate these plots for all products if needed.

In practice, point forecasts are commonly used despite their
limitations, but they do not account for the inherent uncertainty
associated with forecasts. The future is inherently uncertain, and
effective planning requires considering alternative scenarios.
Probabilistic forecasts offer a comprehensive approach by assigning
likelihoods to a range of possible outcomes, recognizing that different
consumption levels may occur with varying probabilities. The primary
purpose of probabilistic forecasting, as illustrated in
Figure~\ref{fig-forecast-density-hstep}, is to quantify and communicate
uncertainty. This figure displays the forecast distribution of
consumption over a 6-month horizon using a density plot. For each month
within the forecast period, a separate distribution is generated. The
plot also includes the point forecast alongside 80\% and 90\% prediction
intervals to illustrate potential variability.

It is important to note that while point forecasts and prediction
intervals can be derived from probabilistic forecasts, the reverse is
not true. A single-point forecast cannot inherently provide the
probabilistic context needed to capture the range of possible outcomes.
Although prediction intervals can indicate a range of potential values,
they do not convey the detailed probabilities of low or high
consumption. This distinction highlights the value of probabilistic
forecasting in supporting informed decision-making by offering a clearer
view of future uncertainties.

\begin{figure}[H]

\centering{

\includegraphics{main_files/figure-pdf/fig-forecast-density-hstep-1.pdf}

}

\caption{\label{fig-forecast-density-hstep}A graphical illustration of
the forecast distribution of a pharmacutical product (i.e.~total
incidence attended) for the SB health board for a horizon of six month.
For each month, we display the point forecast (black point), the
histogram, and 80\% (thick line) and 90\% (thin line) prediction
intervals. It also shows a portion of a historical time series as well
as its fitted values.}

\end{figure}%

\section{Conclusion and future reserach}\label{sec-conclusion}

In this study, we conducted an extensive analysis of pharmaceutical
demand forecasting within the EPSS. Using five years of consumption data
for 33 key pharmaceutical products, along with additional information
gathered through collaboration with domain experts at EPSS, our goal is
to contribute to enhancing the demand forecasting process and emphasize
the importance of integrating domain knowledge into model building. This
step is critical, as the phase of understanding the data and
incorporating valuable contextual information is often overlooked. Too
frequently, forecasting efforts rely solely on consumption data, leading
to building models with little relevance to reality. Our approach
highlights the necessity of including domain insights to construct more
accurate and effective forecasting models.

Our findings highlight the importance of collecting and incorporating
domain knowledge when building forecasting models. We evaluated both
point and probabilistic forecasts using a range of models, from simple
univariate approaches like ARIMA to complex models such as LSTM.
Notably, we demonstrated that employing complex models does not
necessarily lead to more accurate forecasts in the pharmaceutical supply
context. Additionally, our research revealed that newly developed
foundational time series forecasting models have potential
applicability, particularly in environments where advanced analytical
skills and knowledge are limited, as often seen in pharmaceutical supply
chains in developing countries.

To advance forecasting practices further, exploring additional predictor
variables---such as public health campaigns, disease outbreaks, and
economic indicators---could provide valuable insights into demand
dynamics. Improving the granularity of historical consumption data,
through finer temporal intervals and geographic differentiation, also
holds potential for more precise predictions. Additionally, hybrid
forecasting models that combine judgmental point forecasts with
probabilistic machine learning models could enhance predictive
performance. Empowering EPSS staff through training in foundational time
series models, data interpretation, and collection will enable informed
decision-making and strengthen workforce capabilities. Replicating this
study across diverse healthcare settings and conducting comparative
analyses would help validate the adaptability and applicability of the
findings. Additionally, it is recommended that pharmaceutical supply
services systematically collect and maintain detailed records of events
that may influence consumption, alongside consumption data. This
practice is essential for refining demand models and enhancing forecast
accuracy.

\subsection{Practical challenges and
limitations}\label{practical-challenges-and-limitations}

Despite dedicated efforts to engage with domain experts to better
understand the data---and spending significant time over two weeks
collaborating with them---we found that the process of interpreting data
through domain knowledge is complex and presents unique challenges.
These challenges include defining what constitutes domain knowledge and
determining how best to incorporate it to enhance model reliability and
accuracy. Below, we summarize some of these key challenges. One issue
involved capturing accurate information from bin cards and expert
review, particularly during periods of disruption like COVID-19 and
conflicts. For instance, during the COVID-19 pandemic, experts noted a
decrease in demand due to travel restrictions. However, despite the
reduced demand, there were still significant distributions of
pharmaceuticals from the EPSS central to various hubs and health
facilities. This was because, in response to anticipated shortages, a
political decision was made to push products to these facilities before
the travel restrictions took full effect. The rationale was that if the
facilities remained closed, patients would have no access to
medications, necessitating the preemptive stockpiling of essential
pharmaceuticals. This scenario illustrates how policy decisions can
significantly impact supply chain data, making it challenging to model
distribution patterns accurately. Similarly, experts indicated that
conflicts would disrupt the distribution of pharmaceuticals. While
conflict zones did indeed hinder transportation, distribution still
occurred whenever roads were temporarily opened, even amidst on-going
conflict. This created inconsistencies in the data, as periods of halted
distribution were followed by rapid replenishment once access was
restored. Such fluctuations add complexity to modelling the supply chain
from the central EPSS to hubs and health facilities. Other challenges
involved seasonal or periodic activities, such as the fiscal year-end
stock counting. During this time, EPSS temporarily halts transactions to
conduct a full inventory count, which can take anywhere from one to two
months, depending on the circumstances. This inconsistency in the
duration of inventory counts adds a layer of unpredictability to supply
chain modelling.

\subsection{Future reserach}\label{future-reserach}

Following the current research, several promising areas could further
advance knowledge and practical applications in this area:

\begin{itemize}
\item
  Future research could focus on developing intelligent systems that
  incorporate domain-specific knowledge into model building. This would
  involve creating algorithms capable of identifying relevant domain
  insights and integrating them effectively into model training
  processes. Such tools would bridge the gap between purely data-driven
  approaches and expert-knowledge-enhanced modeling, leading to more
  robust and contextually informed forecasts.
\item
  Large Language Models (LLMs) for time series forecasting, like the one
  included in this study, offer transformative opportunities for
  developing countries by making advanced forecasting methods
  accessible, even in contexts with limited expertise or infrastructure.
  However, future research should explore their applicability,
  advantages, limitations, and performance relative to traditional and
  hybrid forecasting methods.
\item
  The use of distribution-based representation of error metrics across
  different rolling origins or forecast horizons, such as boxplots, to
  showcase forecast accuracy provides essential insights into
  variability in accuracy at different point of time in data. Building
  on this, future work could aim at designing methodologies that
  identify diagnostic patterns within the distribution of errors. These
  approaches could suggest model improvements by analyzing error
  distributions and help practitioners to refine models based on error
  behavior.
\item
  While current study focus on monthly aggregate consumption at the
  organizational level, there is potential in exploring more granular
  forecasts. Future research could investigate daily or weekly
  forecasting and examine the impacts at hierarchical levels, such as
  specific sites or health facilities. Investigating how fine-grained
  temporal and hierarchical forecasts influence inventory management and
  operational efficiency could yield valuable insights for reducing
  stockouts, minimizing costs and reducing waste.
\end{itemize}

\section*{Reproducibility}\label{reproducibility}
\addcontentsline{toc}{section}{Reproducibility}

We use R and Python for model development and accuracy evaluation. The
paper is written in Quarto, and all code, data, and files will be shared
in a GitHub repository upon acceptance for publication.


\renewcommand\refname{References}
  \bibliography{references.bib}


\end{document}
